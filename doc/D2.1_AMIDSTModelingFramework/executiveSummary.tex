\section{Executive summary}\label{section:executiveSummary}

The main objective of this document is to design a probabilistic model class (AMIDST modelling framework)  that can efficiently support reasoning in data stream, and in particular, to address the different scenarios encountered in each use-case providers, namely, Daimler AG, Cajamar, and Verdande Technology AS. 

Given the dynamic nature of the problems and the data provided by the different use cases, this deliverable includes not only the static modelling framework proposal, but also and more importantly, a dynamic modelling framework proposal that contains the static version as a particular case. The advantage of considering the dynamic models at this early stage is that all the relevant considerations that mainly arise form a dynamic context are already taken into account.  

To this end, we first try to identify and analyse the family of probability distributions existing in the data provided by each use case. In order to have a clear understanding of this and subsequent analysis in the document, we also include a Section with background on Bayesian networks and different dynamic modelling structures, touch upon state-of-the-art for modelling hybrid distributions (which is more technically reviewed in deliverables D3.1 and 4.2) and various data analysis techniques.

As a result of this data analysis for each use case and the expert knowledge provided, we take into consideration the expressibility of the existing structures to propose, in the first place, particular models for each of the 3 use-case providers.

The combination of these three use-case-tailored static and dynamic models, will allow us to define a preliminary 3-layered AMIDST modelling framework. We argue that this framework will be able to tackle all use cases with different data-driven learning methods, as well as to take advantage of structural adjustments with respect to a particular domain.

It is crucial here to note that the current AMIDST modelling framework should be considered as an initial proposal. The proposal might be updated/extended in the future in the light of evaluation results of the proposed model. The final AMIDST modelling framework will be presented in Deliverable 2.2.


%This document describes the AMIDST modelling framework which consists of the general AMIDST model class as well as the specification of the model classes of the three use-case providers, namely, Daimler AG, Cajamar, and Verdande Technology AS.

%First, the identified preliminary model classes are introduced, for each use-case, based on the requirements analysis (see Deliverable 1.2 \cite{Fer14b}) as well as the preliminary data analysis (including the use of several tools such as sample correlograms, sample partial correlograms, histograms and bivariate contour plots).

%Next, the general AMIDST model class is defined such that it ensures three main requirements: 1) it takes all the preliminary model class characteristics into account; 2) it is applicable to any future, potentially similar, use-cases; and 3) it is scalable, supporting both inference and learning from massive data streams.

%It is crucial here to note that the current AMIDST modelling framework should be considered as an initial proposal. The proposal might be updated/extended in the future, in order to better meet AMIDST requirements and be adapted based on new domain insights. The final AMIDST modelling framework will be presented in Deliverable 2.2.
