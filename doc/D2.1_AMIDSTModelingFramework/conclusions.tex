\section{Ongoing and future work}\label{section:conclusions}

Is is important to emphasize that the modeling framework described in the present document represent a first
proposal. Thus, in light of expected future feedback from the use case providers and the experiences and results obtained from the design of
the inference and learning algorithms, the modeling framework may be adapted accordingly. For example, the modelling
framework is now assuming a first-order Markov assumption. One could envision the need for modeling higher order Markov assumptions, if we find that this first-order assumption severely limits the modelling capacities of this framework in some of the application scenarios. 

We plan indeed to study the potential limitations of the final proposed modelling framework. This framework needs to handle extremely large data streams, and hence, it should be efficient enough to do that, possibly at the expense of a certain loss of expressibility power. A clear identification of the framework's strengths and limitations will be beneficial for a better understanding of the chosen trade-offs between expressibility and computational complexity.

Finally, as already mentioned above,  we are now working on inference (WP3) and learning algorithms (WP4) that make this modelling framework applicable to scenarios dealing with massive data streams. These developments might also have an impact in the current framework. Special emphasis will be given to the study of bounded and unbounded time horizon models as way to trade-off between model expressibility and model complexity. This kind of analysis will be considered in combination with the proposed inference and learning algorithms.

The outcome of all these analyses will be included in Deliverable 2.2.
