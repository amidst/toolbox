\section{The AMIDST modelling framework}\label{section:AMIDSTmodelClass}

One of the main goals of the AMIDST project is the definition of a general model class with the following characteristics: 

\begin{itemize}
\item \textbf{Feature 1:} it should be applicable to the three considered use-cases, i.e., Daimler, Cajamar and Verdande;

\item \textbf{Feature 2:}  it should be general enough to be applicable to any future, potentially similar, use-cases;

\item \textbf{Feature 3:} it should allow the integration of domain/expert knowledge about the structure of the process being modelled, and

\item \textbf{Feature 4:} it should be scalable, supporting both inference and learning from massive data streams.

\end{itemize}


%When designing this model class, we also tried to define a model family that eases as much as possible the integration of domain/expert knowledge, which is a common place in the different application scenarios discussed in Section \ref{Section:PreliminaryModels}.

The AMIDST modelling framework was designed with the goal of meeting these four above requirements. It is introduced in the following two sections. In Section \ref{StaticFramework}, we firstly discuss the static modelling framework that covers the static models presented for some of the application scenarios, but whose main aim is to be extended to a dynamic context. In Section \ref{GeneralModelClass}, we introduce the general AMIDST modelling framework as a dynamic extension of the previous static modelling framework and we show how this modelling framework can be instantiated to solve each of the application scenarios of the different use cases, as proof of concept of the generality of this modelling framework. The graphical notation introduced in Section \ref{SubSection:GraphicalNotation} will be employed when describing these probabilistic graphical models. Finally, a summary with a higher level view of the AMIDST modelling framework is included in Section \ref{summaryAMIDSTModels}.

%Taking these different characteristics into account and using the graphical notation introduced in Section \ref{SubSection:GraphicalNotation}, the general AMIDST model class, as well as its specific instantiations to each use case, are introduced and discussed in Section \ref{GeneralModelClass}. A summary with a higher level view of the AMIDST model class is finally included in Section \ref{summaryAMIDSTModels}.

%-----------------------------------------------------------------------------------------------------------------------------------------
\subsection{The static modelling framework}\label{StaticFramework}
%-----------------------------------------------------------------------------------------------------------------------------------------

Figure \ref{Figure:StaticModellingFramework} visually shows the proposed static modelling framework. This general model is the result of combining the high-level structure of the static models described in some of the application scenarios discussed in Section \ref{Section:PreliminaryModels}. This static modelling framework has been designed having in mind that it should be later on easily extended to a dynamic context. As it can be seen, subnetworks have been used to group variables with similar properties, so that the commonalities between all models are taken into account. 

\begin{figure}[ht!]
\begin{center}
\includegraphics[scale=0.4]{./figures/StaticModellingFramework}
\caption{\label{Figure:StaticModellingFramework} The static modelling framework.}
\end{center}
\end{figure}


The first main characteristic of this static model is that all the modelling variables are structured in three different layers, each one with clear different semantics. We believe that this layered division of the modelling variables can facilitate the integration of domain/expert knowledge. These three layers can be described as follows:


\begin{itemize}

\item \textbf{Control/class layer:}  This upper layer can be instantiated as a target variable for classification models. Alternatively, it can  act as a ``control'' layer, by containing specific variables in the systems that directly or causally influence the rest of variables of the model. This ``control'' meaning will gain significance in the dynamic context.

The layer is represented by a subnetwork including either continuous or discrete observed variables. The links from the continuous observed variables to the state subnetworks in the \textit{Hidden layer} (i.e., the set of ``State 1'' and ``State 2'' variables) will be possibly handled with logit and probit functions.

\item \textbf{Hidden layer:}  This layer attempts to capture those parts of the system/process that can not be observed or measured, but that our domain/expert knowledge knows they are part of the system/process and need to be modelled. Moreover, we can also use this layer as a way to capture the complex conditional dependencies and distributions (e.g. mixture of Gaussians) that might be present between the variables in the observable layer. 

This layer is structured as a set of interconnected discrete and continuous hidden subnetworks, for which only links from discrete to continuous subnetworks are allowed. We will use the term \textit{state} variables to refer to the set of discrete hidden variables in both ``State 1'' and ``State 2'' subnetworks, and the term \textit{latent}\footnote{The term \textit{latent} for variables is generally used in statistics to refer to hidden variables as opposed to observed ones. However, we will use it here to exclusively refer to continuous hidden variables.} variables to refer to the set of continuous hidden variables in the ``Latent'' subnetwork. The distinction between the subnetworks ``State 1'' and ``State 2'' will be better understood in the dynamic context. 

\item \textbf{Observable layer:} This layer accounts for those variables modelling the part of the system/process that can be observed/measured. These observations are conditioned by the variables in the above two layers. 

This layer is represented by a subnetwork including both discrete and continuous observed variables. These variables can in principle be interconnected but, in our different use cases, only links from discrete to continuous nodes are required. However, in general, we would not need to restrict the direction of the links between the variables in this layer.

\end{itemize} 


According to the discussions in Section \ref{Section:PreliminaryModels} about the distribution families governing the use cases, we designed a modelling framework that can be parametrized in alternative ways to improve the expressibility and applicability to different problem domains. This parametrization is also influenced by the possible impact it can have in the inference and learning algorithms. In general, this modelling framework falls inside the conditional linear Gaussian framework, so the conditional probability distributions are parameterized as detailed in Section \ref{SubSection:HybridBNs}. But this modelling framework might also contain instantiations which are not covered by this general framework. More precisely, when the top layer class is instantiated to  a continuous subnetwork, then the conditional probabilities of ``State 1'' and ``State 2'' will be instantiated with logistic or probit distributions, because we encounter continuous parents with discrete children. Additionally, we also envision the possibility of introducing in the observable layer, the use of probability distributions belonging to the MoTBF family \cite{Langseth12}, in order to extend the modelling capacities of this modelling framework.

Full details about how this framework can be instantiated to solve the different application scenarios will be given in the next section. But, just as an illustrative example, we show in Figure \ref{Figure:StaticModellingInstantiations} how this static modelling framework can be instantiated to the static models presented for Daimler (see Section \ref{Section:Daimler:EarlyRecognition}) and CajaMar (see Section \ref{SubSection:Predicting}). In the case of Verdande, no static models has been considered. 


\begin{figure}[ht!]
\begin{center}
\begin{tabular}{cc}
\includegraphics[scale=0.4]{./figures/DaimlerStaticModelling}
&
\includegraphics[scale=0.4]{./figures/CajaMarStaticModelling}
\\Daimler Static Model & CajaMar Static Model \\
\end{tabular}
\caption{\label{Figure:StaticModellingInstantiations} Illustrative example with instantiations of the static modelling framework for the Daimler and CajaMar static models, respectively.}
\end{center}
\end{figure}
%-----------------------------------------------------------------------------------------------------------------------------------------
\subsection{The general AMIDST modelling framework}\label{GeneralModelClass}
%-----------------------------------------------------------------------------------------------------------------------------------------

Figure \ref{Figure:AMIDSTModelClass} shows the proposed general AMIDST modelling framework. This general framework is the result of the temporal extension of the static modelling framework presented in the above section. It is also the result of combining all the dynamic models that have been previously defined for the different use cases. 

\begin{figure}[ht!]
\begin{center}
\includegraphics[scale=0.465]{./figures/AMIDSTModelClass}
\caption{\label{Figure:AMIDSTModelClass} The AMIDST modelling framework.}
\end{center}
\end{figure}

The general AMIDST modelling framework can be seen as a 2T-DBN (see Section \ref{SubSubSection:2DBNs}), i.e., it satisfies the first order Markov property and the stationarity assumption. However, this model is not as general as a 2T-DBN and presents a more restricted internal structure that seeks to offer a good trade-off between expressiveness and efficiency. 

We should point out that the static modelling framework described in Section \ref{StaticFramework} can be seen a simple instantiation of this general model when the temporal dependencies are obviated, that is, when only one time slice of the 2T-DBN is considered. 

One of the main characteristics of this temporal extension is that the inter-time slices dependencies are only allowed between variables in the same layer. At this stage, we have not found any strong argument to allow for more complex instantiations. The three different layers maintain the same semantics as in the static case, although now they can additionally be interpreted from a dynamic point of view. We briefly comment which are the main consequences: 

\begin{itemize}

\item \textbf{Control/class layer:} Now, this layer can be instantiated to support dynamic classification, i.e. we aim to predict a temporal sequence of labels rather than a single label. If it is instantiated as a control layer, it can be used to model the transition probabilities between the variables of the hidden layer. This can be seen as a modelling trick to smooth out the stationary assumption and hence, wider the applicability of the framework.

\item \textbf{Hidden layer:}  In the dynamic case, hidden variables in this layer can be temporally connected. In that way, it is easy to see that our modelling framework subsumes those basic dynamic models commented in Section \ref{SubSection:DBNs}. As in the static case, these variables aim to capture those dynamic parts of the system that can not be observed but are relevant for modelling the dynamic system/process. State and latent variables that are connected over time can be used to model a process that is evolving and changing over time.

As can be seen in Figure \ref{Figure:AMIDSTModelClass}, this layer can contain state variables that are not connected over time (i.e. variables in ``State 2'' subnetwork), and that could be used, for instance, to model mixture of Gaussian distributions over the observed continuous variables in the \textit{Observable layer}. 

Recall that links from latent variables to state variables are not allowed, neither in the same time step nor between consecutive ones.

\item \textbf{Observable layer:}  Observed variables in this layer have now the possibility of being connected over time. Again, there is no restriction on the temporal links between continuous and discrete variables. 

\end{itemize} 



In the following sections, we will show how this modelling framework satisfies the aforementioned \textbf{Feature 1}, by describing how each particular use case fits in this restricted 2T-DBN AMIDST model. This instantiation process of the general AMIDST model to the three use-cases (all of them with different application scenarios) can also be seen as a practical example demonstrating the general applicability of the resulting AMIDST modelling framework. This hence gives arguments in favour of \textbf{Feature 2} and shows how the AMIDST modelling framework can be potentially instantiated to other future use cases. In our opinion, these different instantiations also show how the layer and components based structure of this modelling framework  might also ease the process of building other future models using domain/expert knowledge, giving also arguments in favor of \textbf{Feature 3}. Finally, the remaining task is related to \textbf{Feature 4} and consists of designing inference and learning algorithms that allow the application of the AMIDST modelling framework to massive data streams. In that sense, we might expect that during this process, we might modify some of the elements and assumptions of this modelling framework if this is needed to meet the computational and modelling performance required by the different use-case providers. 



%-----------------------------------------------------------------------------------------------------------------------------------------
\subsubsection{Daimler model class}\label{daimlerAMIDSTModels}
%---------------------------------------------------------------------


Daimler's model class has been previously displayed in Figure \ref{Figure:daimlerLEdynGeneric}. Taking the general AMIDST modelling framework into account, Daimler's model class can be reinterpreted, according to Figure \ref{Figure:AMIDSTModelClassDaimler}, as follows:

\begin{figure}[ht!]
\begin{center}
\includegraphics[scale=0.39]{./figures/AMIDSTModelClassDaimler}
\caption{\label{Figure:AMIDSTModelClassDaimler} AMIDST modelling framework - Daimler}
\end{center}
\end{figure}

\begin{itemize}

\item \textbf{Control/class layer}: It is not used in this domain.

\item \textbf{Hidden layer}: The two sets of state variables are required in this layer. 

The lower set of state variables (i.e., ``State 2'' subnetwork in Figure \ref{Figure:AMIDSTModelClass}) encodes a polytree \cite{JensenNielsen2007} for the hierarchy of the Hypothesis. This polytree structure, not explicitly encoded in the model, can indeed be exploited during inference.

Connection through time slices is only made at the top level state variables (i.e., ``State 1'' subnetwork in Figure \ref{Figure:AMIDSTModelClass}), which corresponds to the signal real values (S\_REAL\footnote{Although these variables are inherently continuous, we notice that they are discretized to avoid the inference problems derived of having continuous parents with discrete children (see Section \ref{Section:DaimlerDynamic}).}). Consequently, the future and past time slices of our 2T-DBN are conditionally independent given S\_REAL variables corresponding to the present time. 

In addition, inside a time slice, the observed continuous and hidden discrete subnetworks are also conditionally independent given S\_REAL variables. In the case that we want to contemplate a possible extension in which the lower level Hypothesis are connected over time, then ``State 1'' subnetwork would contain both the S\_REAL variables and the temporally linked Hypothesis. However, as commented before, this would greatly complicate the inference process. 

\item \textbf{Observable layer:} There are two sets of observed variables in this case. On one hand, we encounter a group of variables representing the measured data (S\_MEAS), along with the variables encoding the sensor noise (S\_SIGMA). The resulting structure is again a polytree, which presents some advantages for the inference process. On the other hand, we have a single discrete node for the manoeuvre Event, which will be the target variable during inference. 

\end{itemize}

%-----------------------------------------------------------------------------------------------------------------------------------------
\subsubsection{Cajamar model class}\label{cajamarAMIDSTModels}
%-----------------------------------------------------------------------------------------------------------------------------------------

Concerning the Cajamar use-case, both application scenarios share the same modelling structures (see Section \ref{Section:CajaMarModels}). The static approach, as discussed above, can be trivially seen as a particular case of the dynamic one. At this point, we focus on the dynamic model, since the general AMIDST modelling framework is primarily a dynamic model class. Thus, the high-level description of Cajamar's dynamic model (previously displayed in Figure \ref{fig:component}) can be reinterpreted according  to Figure \ref{Figure:AMIDSTModelClassCajamar}. 

\begin{figure}[ht!]
\begin{center}
\includegraphics[scale=0.39]{./figures/AMIDSTModelClassCajamar.png}
\caption{\label{Figure:AMIDSTModelClassCajamar} AMIDST modelling framework - Cajamar.}
\end{center}
\end{figure}


Following the layer-wise analysis used above, Cajamar's model is described as follows:
\begin{itemize}

\item \textbf{Control/class layer}: The top node ``Defaulter" in this layer, represents the class variable to categorize a client as defaulter or non-defaulter. There exist temporal links between consecutive time steps to model the dynamic nature of being a defaulter or non-defaulter. Note that this is indeed the target variable in the inference process.

\item \textbf{Hidden layer}: It is not used in this domain.

\item \textbf{Observable layer:} The ``Observed" continuous subnetwork represents information corresponding to the socio-demographic variables, memory variables, financial activity, and past payment behaviour of a client, which, in principle, may or may not be connected over time. Moreover, the ``Indicator" discrete subnetwork includes the set of indicator discrete variables denoted as $\delta_{X_t}$. These indicator variables are used for modelling situations in which some variables in the ``Observed" continuous subnetwork have a large number of zeros.

\end{itemize}


%-----------------------------------------------------------------------------------------------------------------------------------------
\subsubsection{Verdande model class}\label{verdandeAMIDSTModels}
%-----------------------------------------------------------------------------------------------------------------------------------------

Figure \ref{Figure:AMIDSTModelClassVerdande} shows how the general AMIDST modelling framework is instantiated in the case of Verdande. This instantiated model encompasses the models of the three application scenarios previously discussed in Section \ref{Section:VerdandeModels} and depicted in Figures \ref{Figure:VTScenario1}, \ref{Figure:VTScenario2} and \ref{Figure:VTScenario3}.

\begin{figure}[ht!]
\begin{center}
\includegraphics[scale=0.39]{./figures/AMIDSTModelClassVerdande}
\caption{\label{Figure:AMIDSTModelClassVerdande} AMIDST modelling framework - Verdande.}
\end{center}
\end{figure}

Verdande's model class can then be described as follows:

\begin{itemize}
\item \textbf{Control/class layer}:  This layer includes a subnetwork modelling the set of observed Control variables. In the three application scenarios, the role of these Control variables is to condition the transition probability of the state variables and also, to ensure the modelling of non-stationary transition probabilities.  

\item \textbf{Hidden layer}: This layer is directly instantiated from the general AMIDST modelling framework. However, there are some differences when applied to each particular application scenario:

\begin{itemize}
\item For the first application scenario (see Figure \ref{Figure:VTScenario1}):  ``State 2'' is not needed and ``State 1'' is instantiated to the ``Normal/Abnormal'' state variable. 

\item For the second application scenario (see Figure \ref{Figure:VTScenario2}): ``State 1'' is not needed and ``State 2'' is instantiated to a single multinomial variable whose role is to model mixture of Gaussians at the leaves. 

\item For the third application scenario (see Figure \ref{Figure:VTScenario3}): ``State 2'' is not needed and ``State 1'' is instantiated to a subnetwork containing the ``FormationNo'' and ``Switch'' variables.
\end{itemize}

\item \textbf{Observable layer:} For both application scenarios 1 and 3, this layer is instantiated as a single response variable; while for the application scenario 2, it may be instantiated as a set of (possibly interconnected) response variables. 
\end{itemize}

%-----------------------------------------------------------------------------------------------------------------------------------------
\subsection{Summary}\label{summaryAMIDSTModels}
%-----------------------------------------------------------------------------------------------------------------------------------------

As it has been shown in the previous sections, the proposed general AMIDST modelling framework of Figure \ref{Figure:AMIDSTModelClass} encompasses the different application scenarios of the three use cases, both for static and dynamic contexts. These three use cases come from very different domains: automotive, finance and oil-drilling. In our opinion, these are strong arguments in favour of the generality and applicability of this model class. 

A higher level view of our proposed AMIDST modelling framework is displayed in Figure \ref{Figure:AMIDSTModelClassHighLevel}. As commented before, it shows how the AMIDST modelling framework could actually be seen as a ``restricted" 2T-DBN. It is restricted in three different ways. Firstly, all the nodes are structured in three layers, each one with clear semantics while modelling: Broadly speaking, the \textit{control/class layer} either represents control variables that may affect the process homogeneity/stationarity, or represents the class variable in classification tasks; the \textit{hidden layer} includes a sufficiently rich set of hidden variables to capture the process dynamics and enhance the modelling capabilities; and the \textit{observable layer} encompasses the observables variables such as sensor measurements or predictive attributes. The second restriction states that, as opposed to general 2T-DBNs, the variables in this model class can only be temporally linked to other variables in the same layer. And, finally, the third and last restriction  states that continuous hidden variables cannot point to discrete (hidden or observed) nodes. This last constraint implies that our inference algorithms will not have to deal with the hard and open problem of computing posterior distributions or marginalize over continuous variables which are parents of discrete children nodes. So, in most of the cases, the network can be parameterized using the conditional linear Gaussian framework. And in the particular cases for which this is not possible, because the instantiated model contains continuous parents with discrete children, we will encounter that the continuous nodes are always observed. 

\begin{figure}[ht!]
\begin{center}
\includegraphics[scale=0.4]{./figures/AMIDSTModelClassGeneral}
\caption{\label{Figure:AMIDSTModelClassHighLevel} The high-level AMIDST modelling framework}
\end{center}
\end{figure}



