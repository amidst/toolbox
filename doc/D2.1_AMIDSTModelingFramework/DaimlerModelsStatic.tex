\section{Preliminary Models}
\subsection{Daimler Models}

\subsubsection{Introduction}

Daimler use-case is based on two application scenarios \cite{Fer14}: i) early recognition of a lane change manoeuvre; and ii) earlier prediction of the need for a lane change based on relative dynamics between two vehicles driving in the same lane at different speeds. 

The first scenario has been previously addressed by Daimler \cite{kasper2012object}. The main result of this previous work was a static object oriented Bayesian network \cite{koller1997object} able to detect a manoeuvre 0.6s before execution. The goal now is to extend the prediction horizon for manoeuvre recognition at least 1-2 seconds to further improve the quality of the on board adaptive cruise control. As we will explain later, this improvement is expected to be achieved by a dynamic extension of this previously proposed static model. We built on this previously proposed static model for two main reasons: although with a limit prediction horizon, this static model has proven to be very robust for this task and it is considered to be the gold-standard for this problem in Daimler; the developed models are expected to be integrated in a ECU \cite{Fer14}, and the advances made about this respect for the static model \cite{Weidl2014} can be exploited during the integration of their dynamic counterparts. 



\subsubsection*{Description of Application Scenario 1}

The basic settings of this application scenario are as follows. Let us suppose we are driving our car, which will be referred to as the EGO vehicle, in a highway. This EGO vehicle is equipped with a video camera, radar and some on-board sensors.  Using the data provided by these sensors, the challenge consists on making an early recognition of a manoeuvre either by the EGO or another relevant car in the traffic scene (OBJ). In total, the system is expected to recognise the following set of manoeuvres (a visual description of them is given below in Figure \ref{Figure:DaimlerManeuvers}):
\begin{enumerate}
\item \textbf{Object-CutIn}: A vehicle is moving to the lane where the EGO vehicle is placed.
\item \textbf{Object-CutOut}:  A vehicle that was driving in front of the EGO is leaving the EGO's lane.
\item \textbf{Object-Follow}: There is no lane change. The EGO is driving and there is some other vehicle in front.
\item \textbf{Lane-Follow}: There is no lane change. The EGO is driving and there are no other vehicles in front.
\item \textbf{EGO-CutIn}: The EGO vehicle is moving right-direction to a new lane already occupied by another vehicle. 
\item \textbf{EGO-CutOut}: The EGO vehicle is leaving left-direction the lane where it was driving.
\end{enumerate}

\begin{figure}
\begin{center}
\includegraphics[scale=0.4]{./figures/DaimlerManeuvers}
\caption{\label{Figure:DaimlerManeuvers}Different maneuvers which should be identified by the AMIDST system.  Red blocks represent the EGO vehicle and blue blocks represent other vehicles in the scene.
}
\end{center}
\end{figure}

Instead of working with the raw data from the video, radar and on-board sensors, the manoeuvre recognition system uses the so-called ``object data'', which contains ``high level'' representations or features describing the ``traffic scene'' such as EGO's speed, distance between EGO and another vehicle in front, etc.  
\begin{figure}
\begin{center}
\includegraphics[scale=0.35]{./figures/DaimlerDataFlow}
\caption{\label{Figure:DaimlerDataFlow} Daimler's Data Flow.}
\end{center}
\end{figure}

Figure \ref{Figure:DaimlerDataFlow} contains a visual description of the current data flow used to create this ``object data''.  As can be seen in this figure, in a first step the raw data coming from the video, radar and sensors is preprocessed. In a second step this preprocessed data is fused and the high-level or ``object data'' describing the traffic scene is obtained. 

As commented before, using the resulting ``object data'', Daimler has developed a probabilistic graphical model \cite{kasper2012object} which is able to recognize an ongoing manoeuvre around 0.6 seconds before the manoeuvre really takes place. This probabilistic approach is based on modelling the problem in different layers as shown in Figure \ref{Figure:DaimlerHierarchicalModelling}.



\begin{figure}
\begin{center}
\includegraphics[scale=0.58]{./figures/DaimlerHierarchicalModelling}
\caption{\label{Figure:DaimlerHierarchicalModelling} Hierarchical layers for the recognition of driving manoeuvres.}
\end{center}
\end{figure}

The sensor data is modelled in the first step. Using this layer, a new layer is created on top with the goal of detecting a lane change behaviour. The detection of a lane change behaviour allows the system to model the lane change manoeuvre in a higher layer. Finally, with this information, the system is able to identify the kind of driving manoeuvre which is taking place between a pair of vehicles. 



\subsubsection{The static-OOBN model}

This model will work with the so-called ``object data''. This data mainly consists of a set of measured and/or computed signals or situation-features denoted by $S$ (e.g.. EGO speed, EGO lateral velocity, speed of a car in-front, etc., see \cite{kasper2012object} for further details) describing the traffic scene. The whole modelling is structured in hierarchical layers as detailed in Figure \ref{Figure:DaimlerHierarchicalModelling} and it has been previously implemented \cite{kasper2012object} using an object-oriented Bayesian network (OOBN) \cite{koller1997object}. 


%Even using this high-level features, the modelling problem is very complex. 

%At the same time, the problem contain a lot of structure and can be divided in simpler and similar sub-problems. For example, when deciding whether there is evidence or not that a car is performing a %lateral movement to the right, we can employ two situation-features such as the lateral velocity to the right and the lateral offset w.r.t. the right lane marking of this vehicle to make this decision. But %we will find a quite similar problem when deciding about the lateral movement evidence of the EGO car or any other car, when the only difference that will use another situation-features (e.g. the right %lateral velocity of the EGO) .

The general structure of this OOBN model consists of a number of abstraction levels (see Figure \ref{Figure:DaimlerOOBNAbstraction}): all measured and/or computed signals S\_MEAS are handled with their uncertainties S\_SIGMA. These are represented as object classes at the lowest level (class $S$) of the OOBN. The real values S\_REAL of evidence signals are then used at the next level of the hierarchy to evaluate the hypotheses (class $H$ in Figure \ref{Figure:DaimlerOOBNAbstraction}). The combined evaluation of several hypotheses results in the prediction of events, class E. In our case, the events are modelling traffic manoeuvres of the own and neighbour vehicles.

\begin{figure}
\begin{center}
\includegraphics[scale=0.58]{./figures/DaimlerOOBNAbstraction}
\caption{\label{Figure:DaimlerOOBNAbstraction} Static-OOBN model for the prediction of an event (maneuver) \cite{Weidl2014}.}
\end{center}
\end{figure}

As commented above, the observations characterising a situation are acquired from sensors and computations (see Figure \ref{Figure:DaimlerDataFlow}) and, in consequence, they are regarded as  \textit{measured data}. If the measurement instrument is not functioning properly (due to sensor noise or fault), then the sensor-reading (S\_MEAS) and the real variable (S\_REAL) under measurement need not to be the same. This fact imposes the causal model structure as shown in the first part on Figure \ref{Figure:DaimlerOOBNAbstraction}. The sensor-reading of any measured variable is conditionally dependent on random changes in two variables: real value under measurement (S\_REAL) and sensor fault (S\_SIGMA).

The situation features used for manoeuvre recognition are structured along three main dimensions: lateral evidence (LE), trajectory (TRAJ), and occupancy schedule grid (OCCGRID). They represent the three hypotheses (see Figure \ref{Figure:DaimlerOOBNAbstraction}), which are modelled by the corresponding OOBN-fragments \cite{kasper2012object}. The BN fragment for the hypothesis LE is shown in Figure \ref{Figure:DaimlerLE}. Its conditional probability distribution is represented by a sigmoid (logistic) function. This is used to model the growing probability for the lateral evidence to cross the lane marking, based on the vehicle coming closer to the lane marking (modelled by O\_LAT\_MEAS) and the increase of its lateral velocity (modelled by V\_LAT\_ MEAS).

\begin{figure}
\begin{center}
\includegraphics[scale=0.58]{./figures/DaimlerLE}
\caption{\label{Figure:DaimlerLE} Static BN fragment for the LE hypothesis.}
\end{center}
\end{figure}

The right-hand square on Figure \ref{Figure:DaimlerOOBNAbstraction} abstractly shows how these hypotheses are combined into events, which in our automotive scenario correspond to the different driving manoeuvres: lane follow, lane change (cut-in, cut-out), expressed for ego and surrounding objects\cite{kasper2012object}.

