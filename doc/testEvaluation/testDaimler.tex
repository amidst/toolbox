 \documentclass{article}

\usepackage{times}
\usepackage{graphicx}
\usepackage{latexsym}

\usepackage{bm}
\usepackage{amsbsy}
\usepackage{amsmath}
\usepackage{amsfonts}
\usepackage{amssymb}

\usepackage{subfigure}

\usepackage{theorem}

\theoremstyle{theorem}
\newtheorem{theorem}{Theorem}

\theoremstyle{definition}
\newtheorem{definition}{Definition}
\newtheorem{remark}{Remark}

\newcommand{\bu}[1]{\mathbf{#1}}
\newcommand{\bv}[1]{\bm{#1}}




\title{Practical Considerations for Testing the Daimler Use Case}
%\author{Sigve Hovda \\
%Norwegian University of Science and Technology\\
%Department of Computer and Information Science,
%Trondheim, Norway\\
%sigveh@idi.ntnu.no}
\date{}


\begin{document}
\maketitle

The application scenario is early recognition of lane change manoeuvre.  Below is a very simplistic explanation of how an algorithm can be tested. 

\section{Early recognition of lane change manoeuvre}

A \emph{lane change manoeuvre} is either a EGO cut in, EGO cut out, object cut in and object cut out. Moreover, a \emph{no lane change manoeuvre} is either a object follow or a lane follow.  In the testing regime we have a binary classification problem: lane change or no lane change and the subdivision into the other categories is not a concern regarding testing this particular application scenario.

Daimler has provided us with a number of time series, some including a lane change and some involves only lane follow. The time series for lane change are 15 seconds long involving 10 seconds of data before the lane change and 5 seconds after the lane change. Some of these time series are further chopped into 5 seconds where the last data point is always a lane change. IN the further analysis we can assume that all time series are 5 seconds long ending with a lane change, either because they were like this as we got them from Daimler or we chopped them ourselves. The same goes for lane follow, all time series are 5 seconds long.

The resolution is fixed, but is dependent on which camera is used (typically in the range of 40 - 60 ms).  Missing values are common and a guess is about every second measurement is missing.  There are 93 time series of lane follow and 148 time series of lane change.  The lane change happens always on the last sample. There are no overlap between any of these time series, and they can be seen as independent of each other.

The time series are denoted $\bu{s_i}$, where each vector of measurement at timestep $j \in \{1,2, ... n\}$ is denoted $\bv{s_{i,j}}$. This vector includes physical quantities such as distances, velocities and accelerations.  At every time step $j$, the static and the dynamic Bayesian networks calculates the values  $P(\mbox{Lane Change}_j \,|\,  \bv{s_{i,j}})$ and \\
$P(\mbox{Lane Change}_j \,|\, \{ \bv{s_{i,0}}, \bv{s_{i,1}}, ... \bv{s_{i,j}} \})$, respectively.

It is essential that both models (static and dynamic) predict the probability of whether there is an actual lane change at moment $j$ or not. It is not calculating the probability of there being a lane change between for instance one and two seconds from now.  




The system shall run on these time series and provide us with information about if there is a lane change and if there is a lane change, how many seconds before the actual crossing did it happen.

\subsection*{Calculation of true positives and true negatives}

 In essence the Bayesian network is computing an estimate of $P(\mbox{Lane Change}_j \,|\, \{ \bv{s_{i,0}}, \bv{s_{i,1}}, ... \bv{s_{i,j}} \})$ at every timestep for each timeseries $\bu{s_i}$.  Let $T_1$ and $T_2$ be corresponding to entries in the timeseries that are one and two seconds before the end of the time series.   

In the experiment of detection at least one seconds before lane change, a positive is detected for time series $\bu{s_i}$, when  $P(\mbox{Lane Change}_{T_1} \,|\, \{ \bv{s_{i,0}}, \bv{s_{i,1}}, ... \bv{s_{i,T_1}} \}) \geq C$.  Otherwise a negative is detected.  By doing this on all time series, the number of true positives and true negatives can be counted. Moreover, by allowing the threshold $C$ to vary over all relevant values a ROC curve can be drawn and therefore eventually AUROC can be calculated.

A similar path is followed for detecting two seconds before.  The only difference is that a positive is defined as when  $P(\mbox{Lane Change}_{T_2} \,|\, \{ \bv{s_{i,0}}, \bv{s_{i,1}}, ... \bv{s_{i,T_2}} \}) \geq C$.

\subsection*{Requirements}
Two requirements are needed.  For the first use case scenario it is required that AUROC should be above 0.96 for prediction 1 second before lane crossing and 0.90 for the 2 second prediction.  We will refer to these as AUROC1 and AUROC2.  

\subsection*{Questions: }
\begin{enumerate}
\item Can you go though all the information and make sure that it is correct.
\end{enumerate}


%
%
%\section{Recognition of need for lane change manoeuvre}
%
%It is our understanding that Daimler has not yet provided us with data for this use case scenario.  In order to test this use case scenario, we need a set of time series where there is no need for lane change and a data set where it is actually a need for lane change. 
%
%\subsection*{Questions: }
%\begin{enumerate}
%\item Should the time series for no need for change be time series of object follow where no lane change happens?
%\item Should the time series for need for change be time series of object follow where there is a lane change at the end?
%\item If the answer is yes to both questions, can we follow a similar path to what with did in first use case scenario?
%\end{enumerate}

\bibliographystyle{named}
\bibliography{ijcai13}

\end{document}

