There is a general lack of formal and agreed upon methods for conducting requirement engineering RE for smaller software projects \cite{Ara07,Qui10}.  Moreover, to our knowledge, there exist no publications on small projects where the scope is specified before the project is started.  However, many small software projects are of this type, since the pre-specified scopes may occur as a result of a grant agreements that partially fund the projects.  

In this paper the RE process for the AMIDST (Analysis of MassIve DataSTreams) project is outlined.  AMIDST is a project that is partially funded by the European Union's Seventh Framework Programme for research, technological development and demonstration.  This project has a pre-specified scope, because the grant agreement demands that a requirement engineering process is conducted within a pre-specified time frame and also that the result of the requirement process is in compliance with the Description of Work DoW that is agreed upon.   

The AMIDST RE process is based on selected methodological approaches from existing RE processes, which subsequently have been tailored to the specific characteristics of the AMIDST project.  The characteristics in the AMIDST project are; there exist a pre-specified scope, partners are located at different locations, massive transfer of domain knowledge between partners, one software framework to be used in three different industrial domains and that it is expected that the project focus will be refined.  

In general, a use case driven approach is followed, because functional requirements have been in focus. Another important tool in the AMIDST RE process is the development of a unified and formal template for elicitation to encourage the overall transparency of the process.