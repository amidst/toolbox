\newpage
\subsection{DataInstance}
\label{DataInstance:ID}

\begin{description}
\item[Deadline:] M12
\item[Responsible:] Sigve
\item[Package:] \texttt{eu.amidst.core.database}
\end{description}

%--------------------------------------------------------------------------------------------
\subsubsection*{Description}
%--------------------------------------------------------------------------------------------

DataInstance refers to a row in the data set typically containing a value assignment for each attribute. In terms of code, DataInstance is defined as an interface that is implemented by either a StaticDataInstance or a DynamicDataInstance. The former stores one row (DataRow) of the data set at a time, whereas the second stores two rows of the data set (one for the past and another for the present). 

%--------------------------------------------------------------------------------------------
\subsubsection*{Detailed functionality}
%--------------------------------------------------------------------------------------------

A DynamicDataInstance always has a TimeID and a SequenceID. If this two attributes, or any of the two, are not in the dynamic data set, then they are automatically filled in, incrementally for the TimeID and with a value of $1$ for the SequenceID.

\begin{itemize}
\item StaticDataInstance: only contains a single DataRow.

\item DynamicDataInstance: contains a past and a present DataRow, a TimeID, and a SequenceID. The TimeID attribute in the data set can be used to represent missing samples (stored as DataRowMissing), in which all attribute values are NaN.
\end{itemize}

%--------------------------------------------------------------------------------------------
\subsubsection*{Code example}
%--------------------------------------------------------------------------------------------

